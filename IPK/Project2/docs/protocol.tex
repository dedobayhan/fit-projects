% Configuration
\documentclass[11pt,a4paper]{article}
\usepackage[a4paper, top=2.0cm, bottom=2.0cm, right=2.0cm, left=2.0cm, nohead]{geometry}
\usepackage[utf8]{inputenc}
\usepackage{multicol}

% Disables some warning messages
\sloppy
\hbadness 10000

% Macros
\newcommand{\bs}{$\backslash$}

% Basic document information
\title{IPK - Project 2 - protocol description}
\author{Petr Zemek, xzemek02@stud.fit.vutbr.cz}

\begin{document}

% Page numbering
\pagestyle{empty}

% Header
\noindent
\begin{small}Computer Communications and Networks, 2nd project \\ \today\end{small} \\

% Title
\begin{center}
	\begin{large}\textbf{Client-server communication protocol}\end{large} \\
	\vspace{0.4cm}
	Petr Zemek \\
	\textit{xzemek02@stud.fit.vutbr.cz} \\
	\textit{Faculty of Information Technology, Brno} \\
\end{center}

\noindent
\begin{abstract}
	\noindent
	This paper describes communication protocol between client and server in
	my simple file transfer program. Using this protocol you
	are able to download files from the server, upload files to the server,
	delete files from the server and get file lists from the server, everything
	by using the client part of the program.
\end{abstract}

\subsection*{1 Application part}

Programs communicate by sending and recieving text messages (except binary files transfer).
Each message is in the following form (\texttt{code} means a request or an answer code,
\texttt{data} can be either text or binary (e.g. content of a file)): \\

\par
\begingroup
\leftskip=0.8cm
\noindent
\texttt{code} \texttt{text}\bs \texttt{r}\bs \texttt{n} \\
\texttt{data}
\par
\endgroup

\subsubsection*{1.1 Request codes}
Each request code is followed by a file name (\textit{text} part of the message). \\

\par
\begingroup
\leftskip=0.6cm
\noindent
\begin{tabular}{ll}
	GET &\ request to download selected file from the server \\
	PUT &\ request to upload selected file to the server \\
	DEL &\ request to delete selected file from the server \\
	LST &\ request for the file list (or one particular file) from the server\\
\end{tabular}
\par
\endgroup

\subsubsection*{1.2 Answer codes}
\par
\begingroup
\leftskip=0.8cm
\begin{multicols}{2}
	\noindent
	100 \quad error, requested file was not found \\
	101 \quad error, invalid filename \\
	102 \quad error, can not operate directories \\
	103 \quad error during transmission \\

	\noindent
	104 \quad server error \\
	200 \quad ok, sending file \\
	201 \quad ok, sending only text info
	\\
	\\
\end{multicols}
\par
\endgroup

\vspace{-0.8cm}

\subsection*{2 Transport part}

This has nothing to do with used transport protocol (e.g. TCP/UDP), but it is only the internal
way how to manage message length in the recieving process.
Before the message is send (request/answer), message size is placed before the message itself,
and after recieving this piece of information is removed from the message. \\

\par
\begingroup
\leftskip=0.8cm
\noindent
\texttt{message\_length}$_{(10)}$\bs \texttt{r}\bs \texttt{n} \\
\texttt{message}
\par
\endgroup

\subsection*{3 Conclusion}

The protocol was created to provide an effective way of communication for this program.
The biggest advantage is in the text format, which can be easily understand and use by end user.
It is partially based on the ftp protocol (see rfc959), but it implements only necessary operations.
However, more functions can be added (e.g. download more files in one request).

\end{document}
