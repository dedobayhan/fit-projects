%==============================================================================
% Encoding: utf8
% Project:  ITY - Project 2
% Author:   Petr Zemek, xzemek02@stud.fit.vutbr.cz
%==============================================================================

\documentclass[10pt]{proc}

% Packages
\usepackage[utf8]{inputenc}
\usepackage{czech}
\usepackage{amsmath}
\usepackage{amsthm}
\usepackage{amssymb}
\usepackage{mathrsfs}

% Definitions, algorithms, ...
\newtheoremstyle{ity}{\topsep}{\topsep}{}{}{\bfseries}{}{ }{\thmname{#1}\thmnumber{ #2}\thmnote{ #3}}
\theoremstyle{ity}
\newtheorem{thdef}[subsection]{Definice}
\newtheorem{thalg}[subsection]{Algoritmus}

\begin{document}

% Disable page numbering
\pagestyle{empty}

% Document information
\title{\textbf{Zadání projektu 2 do předmětu ITY} \\ Sazba matematických vzorců a textu}
\author{Petr Zemek}

\maketitle

\section*{Úvod}

Nyní si procvičíme sazbu matematických vzorců, využití matematických prostředí a textových struktur obvyklých
pro technicky zaměřené texty (například viz rovnice (\ref{eq:factorial}) nebo definice \ref{def:cfg} na
straně \pageref{def:cfg}).

\section{Matematický text}

Pro množinu $Q$ označuje $\mathrm{card}(Q)$ kardinalitu $Q$. Konečnou neprázdnou množinu nazýváme
\emph{abeceda}. Pro abecedu $V$ reprezentuje $V^{*}$ volný monoid generovaný touto abecedou $V$ s~operací
konkatenace. Prvek identity ve volném monoidu $V^{*}$ značíme symbolem $\varepsilon$.
Nechť $V^{+} = V^{*} - \{\varepsilon\}$. Algebraicky je tedy $V^{+}$ volná pologrupa generovaná množinou $V$
s~operací konkatenace. Pro $w \in V^{*}$ označuje $|w|$ délku řetězce $w$.
Pro $W \subseteq V$ označuje $\mathrm{occur}(w, W)$ počet výskytů symbolů z~$W$ v~řetězci $w$ a
$\mathrm{sym}(w, i)$ určuje $i$-tý symbol řetězce $w$; například $\mathrm{sym}(abcd, 3) = c$.

\begin{thdef}
	\label{def:cfg}
	\emph{Bezkontextová gramatika} je čtveřice $G = (V, T, P, S)$, kde $V$ je totální abeceda,
	$T \subseteq V$ je abeceda terminálů, $S \in (V - T)$ je startující symbol a $P$ je konečná množina
	\emph{pravidel}	tvaru $q \colon A \rightarrow \alpha$, kde $A \in (V - T)$, $\alpha \in V^{*}$ a $q$ je
	návěští tohoto pravidla. Nechť $N = V - T$ značí abecedu neterminálů.
	Pokud $q \colon A \rightarrow \alpha \in P$, $\gamma, \delta \in V^{*}$, $G$ provádí derivační krok
	z~$\gamma A \delta$ do $\gamma \alpha \delta$ podle pravidla $q \colon A \rightarrow \alpha$, symbolicky
	píšeme $\gamma A \delta \Rightarrow \gamma \alpha \delta \ [q \colon A \rightarrow \alpha]$ nebo
	zjednodušeně $\gamma A \delta \Rightarrow \gamma \alpha \delta$.
	Standardním způsobem definujeme $\Rightarrow^{m}$, kde $m \geq 0$. Dále definujeme tranzitivní uzávěr
	$\Rightarrow^{+}$ a tranzitivně-reflexivní uzávěr $\Rightarrow^{*}$.
\end{thdef}

\begin{thalg}
	Algoritmus pro ověření bezkontextovosti gramatiky.
	Mějme gramatiku $G = (N, T, P, S)$.
	\begin{enumerate}
		\item \label{item:cfalg1} Pro každé pravidlo $p \in P$ proveď test, zda $p$ na levé straně obsahuje
			právě jeden symbol z~$N$.
		\item Pokud všechna pravidla splňují podmínku z~kroku \ref{item:cfalg1}, tak je gramatika $G$
			bezkontextová.
	\end{enumerate}
\end{thalg}

\begin{thdef}
	\emph{Jazyk} definovaný gramatikou $G$ definujeme jako $L(G) = \{w \in T^{*}\ |\ S \Rightarrow^{*} w \}$.
\end{thdef}

\section{Rovnice a odkazy}

$$\sqrt[x^{2}]{y_{0}^{3}} \quad \mathbb{N} = \{1, 2, 3, \dots\} \quad x^{y^{y}} \neq x^{yy} \quad z_{i_{j}}
\not\equiv z_{ij}$$

V~rovnici (\ref{eq:brackets}) jsou využity tři typy závorek s~různou explicitně definovanou velikostí.

\begin{eqnarray}
	\label{eq:factorial} n! & = & \underbrace{n \cdot (n - 1) \cdot \cdots \cdot 2 \cdot 1}_{n} \\
	\label{eq:brackets} - \biggl{\{}\Bigl{[}\bigl{(}a + b\bigr{)} * c\Bigr{]}^{d}\biggr{\}} & = & 0 \\
	\lim_{x \rightarrow 0}\frac{\sin{3x}}{x} & = & 3 \nonumber
\end{eqnarray}

V~této větě vidíme, jak vypadá implicitní vysázení limity $\lim_{n \rightarrow \infty}f(n)$ v~normálním
odstavci textu. Podobně je to i s~dalšími symboly jako $\sum_{1}^{n}$ či $\bigcup_{A \in \mathcal{B}}$.
V~případě vzorce $\lim\limits_{x \rightarrow 0}\frac{\sin{x}}{x} = 1$ jsme si vynutili méně úspornou sazbu.

\begin{eqnarray}
	\int\limits_{a}^{b}f(x)\,\mathrm{d}x & = & - \int_{b}^{a}f(x)\,\mathrm{d}x \\
	\Bigl{(}\sqrt[5]{x^{4}}\Bigr{)}' = \Bigl{(}x^{\frac{4}{5}}\Bigr{)}' & = & \frac{4}{5}x^{-\frac{1}{5}}
		= \frac{4}{5\sqrt[5]{x}} \\
	\overline{\overline{A \vee B}} & = & \overline{\overline{A} \wedge \overline{B}}
\end{eqnarray}

$$\binom{n}{k} =
\left\{
	\begin{array}{ll}
		\frac{n!}{k!(n - k)!} & \quad \mathrm{pro} \ 0 \leq k \leq n \\
		0 & \quad \mathrm{pro} \ k < 0 \ \mathrm{nebo} \ k > n
	\end{array}
\right.$$

\section{Matice}

Pro sázení matic se velmi často používají dvě konstrukce:
(a) konstrukce levé a pravé závorky ($\verb!\left!$, $\verb!\right!$) a
samotná matice hodnot sázených pomocí prostředí $\verb!array!$, jehož základní vlastnosti jsou podobné
prostředí $\verb!tabular!$.

$$\left(
	\begin{array}{c c}
		\hat{a}\heartsuit\aleph & \vec{b} \circ \vec{\imath}\\
		\widehat{c + d} & \vec{b} \\
		\hat{a} & \overrightarrow{AC} \\
		\tilde{\Theta} & \bar{\eta}
	\end{array}
\right)$$

$$\left|
	\begin{array}{c c}
		t & u \\
		v & w
	\end{array}
\right| = tw - uv$$

$$\mathbf{A} = \left\lVert
	\begin{array}{c c c c}
		 a_{11} & a_{12} & \dots & a_{1n} \\
		 a_{21} & a_{22} & \dots & a_{2n} \\
		 \vdots & \vdots & \ddots & \vdots \\
		 a_{m1} & a_{m2} & \dots & a_{mn}
	\end{array}
\right\rVert$$

\section{Závěrem}
V~případě, že budete potřebovat vyjádřit matematickou konstrukci nebo symbol a nebude se Vám dařit jej
nalézt v~samotném \LaTeX{u}, doporučuji prostudovat možnosti balíku maker \AmS-\LaTeX.
Analogická poučka platí obecně pro jakoukoli konstrukci v~\TeX{u}.

\end{document}

% End of file
