%%%%%%%%%%%%%%%%%%%%%%%%%%%%%%%%%%%%%%%%%%%%%%%%%%%%%%%%%%%%%%%%%%%%%%%%%%%%%%%
% Project: GAL 2009
% Authors:
%     Radim Kapavik, xkapav01@stud.fit.vutbr.cz
%     Ondrej Lengal, xlenga00@stud.fit.vutbr.cz
%     Vojtech Storek, xstore02@stud.fit.vutbr.cz
%     Vit Triska, xtrisk01@stud.fit.vutbr.cz
%     Petr Zemek, xzemek02@stud.fit.vutbr.cz
%%%%%%%%%%%%%%%%%%%%%%%%%%%%%%%%%%%%%%%%%%%%%%%%%%%%%%%%%%%%%%%%%%%%%%%%%%%%%%%

\newtheorem{theorem}{Theorem}

\section*{Exercise 6}
\label{sec:Ex6}

\subsection*{Assignment} When an adjacency-matrix representation is used, most
graph algorithms require time $\Omega(V^2)$, but there are some exceptions.
Show that determining whether a directed graph G contains
a \textit{\textbf{universal sink}} --- a vertex with in-degree $|V| - 1$ and
out-degree 0 --- can be determined in time $O(V)$, given an adjacency matrix
for $G$.

\subsection*{Solution}

For a graph $G = (V, E)$, let $n = |V|$ and $m = |E|$.

\textbf{Main idea}

Start in the top left-hand corner of the adjacency-matrix $A$ of $G$ and move
according to the following rules: if you are in a cell with $0$ go right, if
you are in a cell with $1$, go down. If you go down from the bottommost row,
there is not a universal sink in $G$; if you go right from the rightmost
column, the index $i$ of current row is a possible universal sink. In the
latter case, it needs to be verified that the vertex really is a universal sink
by checking that all elements in the $i$-th row are $0$s (i.e.~no edge leaves
the universal sink) and all elements, except the $i$-th one, in the $i$-th
column are $1$s (i.e.~there is an edge from all other vertices leading to the
universal sink).


\textbf{Algorithm}

\begin{algorithm}[H]
  \KwIn{Adjacency-matrix $A = (a_{ij})$ for a directed graph $G = (V, E)$.}
  \KwOut{\textbf{TRUE} if $G$ contains a universal sink, \textbf{FALSE}
  otherwise.}

  \tcp{Initialization}
  $i := 1$\;
  $j := 1$\;

  \tcp{Traverse the matrix}
  \While{$i \le n$ and $j \le n$}{
    \eIf(\tcp*[h]{go to next column}){$a_{ij} = 0$}{
      $j := j + 1$\;
    } ( \tcp*[h]{go to next row}) {
      $i := i + 1$\;
    }
  }

  \tcp{Evaluate the result}
  \eIf{$i > n$}{
    \Return \textbf{FALSE}\;
  } {
    \tcp{Check whether the $i$-th vertex really is a universal sink}
    \For{$k := 1$ \KwTo $n$}{
      \If(\tcp*[h]{edge leading from universal sink}){$a_{ik} = 1$}{
        \Return \textbf{FALSE}\;
      }

      \If{$k \ne i$}{
        \If(\tcp*[h]{no edge leading to universal sink}){$a_{ki} = 0$}{
          \Return \textbf{FALSE}\;
        }
      }
    }
    \Return \textbf{TRUE}\;
  }
\end{algorithm}

\textbf{Complexity analysis}

The initialization part of the algorithm is done in $O(1)$. The body of matrix
traversal loop ensures that either $i$ or $j$ is incremented by one in every
iteration. Due to the terminating condition of the loop ($i \le n$ and $j \le
n$), the loop body is done at most $n + n = 2n$ times. Verification that the
found vertex (if any) is a universal sink takes $n$ steps (while $k$ iterates
from $1$ to $n$). The resulting time complexity is therefore $O(1 + 2n + n)
= O(n)$.

\textbf{Correctness proof}


\begin{theorem}
The algorithm always terminates.
\end{theorem}

\begin{proof}
According to the \textit{Complexity analysis}, the time complexity of the
algorithm is $O(n)$ $\to$ the algorithm always terminates in $O(n)$ time.
\end{proof}

\pagebreak
\begin{theorem}\label{theorem_correct}
The algorithm terminates with \textbf{TRUE} $\Leftrightarrow$ there is
a universal sink in $G$.
\end{theorem}

\begin{proof}
Assume without loss of generality that $V = \{1, 2, \dots, n\}$. Let us
decompose the theorem into two cases:

\begin{itemize}

  \item[``$\Rightarrow$'':] The algorithm terminates with \textbf{TRUE} only
  after it verifies that vertex $i$ really is a universal sink ($(\forall j \in
  V : a_{ij} = 0) \wedge (\forall j \in V \setminus \{i\} : a_{ji} = 1)$).

  \item[``$\Leftarrow$'':] (Proof by contradiction.) Assume that there is a
  universal sink $u$ in $G = (V, E)$ and the algorithm terminates with
  \textbf{FALSE}. According to the algorithm, this can happen in two cases:

  \begin{enumerate}

    \item At the end of matrix traversal $i > n$. Because $i$ only changes by
    1, it must have been assigned all values from $\{1, 2, \dots, n\}$,
    therefore at one point being assigned $u$. However, because the $u$-th row
    contains all $0$s ($\forall x \in V : a_{ux} = 0$), according to the
    condition in matrix traversal loop, upon $i$ reaching $u$, $j$ must have
    been incremented until it reached the rightmost column of the matrix ($j
    > n$). Therefore $i > n$ cannot happen.

    \item At the end of matrix traversal $j > n$ but $i$ is not the universal
    sink, let us denote it as $z$ ($z \ne u$). Two cases are possible:

    \begin{enumerate}

      \item $z > u$. This means that $i$ must have been assigned all values
      from $\{1, 2, \dots, z\}$, including $u$ (because $z > u$). However,
      because the $u$-th row contains all $0$s ($\forall x \in V : a_{ux}
      = 0$), according to the condition in matrix traversal loop, upon $i$
      reaching $u$, $j$ must have been incremented until it reached the
      rightmost column of the matrix ($j > n$) terminating the loop with $i
      = u$. Because $z > u$, this cannot happen.

      \item $z < u$. This means that $j$ must have been assigned all values
      from $\{1, 2, \dots, n\}$, including $u$. Upon $j$ reaching $u$, because
      the $u$-th column of the matrix contains all $1$s except the $u$-th
      element ($\forall j \in V \setminus \{u\} : a_{ju} = 1$), according to
      the condition in matrix traversal loop, $i$ had to be incremented until
      it reached $u$. Due to the fact that the $u$-th row contains all $0$s
      ($\forall x \in V : a_{ux} = 0$), according to the condition in matrix
      traversal loop, $j$ must have been incremented until it reached the
      rightmost column of the matrix ($j > n$) terminating the loop with $i
      = u$. Because $z < u$, this cannot happen.

    \end{enumerate}

  \end{enumerate}

  All cases contradict.

\end{itemize}

\end{proof}

%%%%%%%%%%%%%%%%%%%%%%%%%%%%%%%%%%%%%%%%%%%%%%%%%%%%%%%%%%%%%%%%%%%%%%%%%%%%%%%
% vim: syntax=tex
%%%%%%%%%%%%%%%%%%%%%%%%%%%%%%%%%%%%%%%%%%%%%%%%%%%%%%%%%%%%%%%%%%%%%%%%%%%%%%%
