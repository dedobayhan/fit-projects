%%%%%%%%%%%%%%%%%%%%%%%%%%%%%%%%%%%%%%%%%%%%%%%%%%%%%%%%%%%%%%%%%%%%%%%%%%%%%%%
% Project: GAL 2009
% Authors:
%     Radim Kapavik, xkapav01@stud.fit.vutbr.cz
%     Ondrej Lengal, xlenga00@stud.fit.vutbr.cz
%     Vojtech Storek, xstore02@stud.fit.vutbr.cz
%     Vit Triska, xtrisk01@stud.fit.vutbr.cz
%     Petr Zemek, xzemek02@stud.fit.vutbr.cz
%%%%%%%%%%%%%%%%%%%%%%%%%%%%%%%%%%%%%%%%%%%%%%%%%%%%%%%%%%%%%%%%%%%%%%%%%%%%%%%

\section*{Exercise 7}
\label{sec:Ex7}

\subsection*{Assignment}
The \textit{\textbf{incidence matrix}} of a directed graph $G = (V, E)$ is
a $|V| \times |E|$ matrix $B = (b_{ij})$ such that

$$
b_{ij} =
  \left\{
  \begin{array}{ll}
    -1 & \mbox{if edge $j$ leaves vertex $i$,} \\
     1 & \mbox{if edge $j$ enters vertex $i$,} \\
     0 & \mbox{otherwise}.
  \end{array}
  \right.
$$

Describe what the entries of the matrix product $B B^{T}$ represent, where
$B^{T}$ is the transpose of $B$.

\subsection*{Solution}

For a graph $G = (V, E)$, let $n = |V|$ and $m = |E|$.

$$
B =
\left[
\begin{array}{*{5}{c}}
  b_{11} & b_{12} & b_{13} & \cdots & b_{1m} \\
  b_{21} & b_{22} & b_{23} & \cdots & b_{2m} \\
  b_{31} & b_{32} & b_{33} & \cdots & b_{3m} \\
  \vdots & \vdots & \vdots & \ddots & \vdots \\
  b_{n1} & b_{n2} & b_{n3} & \cdots & b_{nm} \\
\end{array}
\right]
\qquad
B^{T} =
\left[
\begin{array}{*{5}{c}}
  b^{T}_{11} & b^{T}_{12} & b^{T}_{13} & \cdots & b^{T}_{1n} \\
  b^{T}_{21} & b^{T}_{22} & b^{T}_{23} & \cdots & b^{T}_{2n} \\
  b^{T}_{31} & b^{T}_{32} & b^{T}_{33} & \cdots & b^{T}_{3n} \\
  \vdots & \vdots & \vdots & \ddots & \vdots \\
  b^{T}_{m1} & b^{T}_{m2} & b^{T}_{m3} & \cdots & b^{T}_{mn} \\
\end{array}
\right]
\qquad
\mbox{where $b^{T}_{ij} = b_{ji}$}
$$

It holds that

$$
B B^{T} = C =
\left[
\begin{array}{*{5}{c}}
  c_{11} & c_{12} & c_{13} & \cdots & c_{1n} \\
  c_{21} & c_{22} & c_{23} & \cdots & c_{2n} \\
  c_{31} & c_{32} & c_{33} & \cdots & c_{3n} \\
  \vdots & \vdots & \vdots & \ddots & \vdots \\
  c_{n1} & c_{n2} & c_{n3} & \cdots & c_{nn} \\
\end{array}
\right]
$$

where

$$
c_{ij} = b_{i1} b^{T}_{1j} + b_{i2} b^{T}_{2j} + b_{i3} b^{T}_{3j} + \dots
+ b_{im} b^{T}_{mj} =
\sum_{k=1}^{m}(b_{ik} b^{T}_{kj})
$$

However, because $b^{T}_{ij} = b_{ji}$, this can be modified to

$$
c_{ij} = \sum_{k=1}^{m}(b_{ik} b_{jk})
$$

Let us study two distinct cases:

\begin{enumerate}

  \item[$i = j$:] (elements of the main diagonal) In this case

  $$
  c_{ij} = c_{ii} = \sum_{k=1}^{m}(b_{ik} b_{ik}) = \sum_{k=1}^{m} b_{ik}^{2}
  $$

  Because $b_{ik} \in \{-1, 1, 0\}$, it holds that $b_{ik}^{2} \in \{1, 0\}$
  ($(-1)^{2} = 1$, $1^{2} = 1$, $0^{2} = 0$). Therefore,

  $$
  b_{ij}^2 =
  \left\{
  \begin{array}{ll}
     1 & \mbox{if edge $j$ enters or leaves vertex $i$,} \\
     0 & \mbox{otherwise}.
  \end{array}
  \right.
  $$

  This means that $c_{ii} = \sum_{k=1}^{m} b_{ik}^{2}$ is the number of edges
  entering or leaving vertex $i$ (the degree of vertex $i$).


  \item[$i \ne j$:] for this case, it holds that $(b_{ik} b_{jk}) \in \{-1,
  0\}$ (it cannot happen that $b_{ik} = 1$ and $b_{jk} = 1$ for some $k \in
  \{1, 2, \dots, m\}$, because that would mean that edge $k$ leaves two
  distinct vertices (as $i \ne j$). $(b_{ik} b_{jk}) = -1$ iff there is an edge
  from vertex $i$ to vertex $j$ (or vice versa), $0$ otherwise. This means that
  $-c_{ij} = -\sum_{k=1}^{m}(b_{ik} b_{jk})$ is the total number of edges
  between vertices $i$ and $j$.

\end{enumerate}


%%%%%%%%%%%%%%%%%%%%%%%%%%%%%%%%%%%%%%%%%%%%%%%%%%%%%%%%%%%%%%%%%%%%%%%%%%%%%%%
% vim: syntax=tex
%%%%%%%%%%%%%%%%%%%%%%%%%%%%%%%%%%%%%%%%%%%%%%%%%%%%%%%%%%%%%%%%%%%%%%%%%%%%%%%
