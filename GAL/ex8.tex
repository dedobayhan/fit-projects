%%%%%%%%%%%%%%%%%%%%%%%%%%%%%%%%%%%%%%%%%%%%%%%%%%%%%%%%%%%%%%%%%%%%%%%%%%%%%%%
% Project: GAL 2009
% Authors:
%     Radim Kapavik, xkapav01@stud.fit.vutbr.cz
%     Ondrej Lengal, xlenga00@stud.fit.vutbr.cz
%     Vojtech Storek, xstore02@stud.fit.vutbr.cz
%     Vit Triska, xtrisk01@stud.fit.vutbr.cz
%     Petr Zemek, xzemek02@stud.fit.vutbr.cz
%%%%%%%%%%%%%%%%%%%%%%%%%%%%%%%%%%%%%%%%%%%%%%%%%%%%%%%%%%%%%%%%%%%%%%%%%%%%%%%

\section*{Exercise 8}
\label{sec:Ex8}

\subsection*{Assignment}

Suppose that instead of a linked list, each array entry $Adj[u]$ is a hash
table containing the vertices $v$ for which $(u, v) \in E$. If all edge lookups
are equally likely, what is the expected time to determine whether an edge is
in the graph? What disadvantages does this scheme have? Suggest an alternate
data structure for each edge list that solves these problems. Does your
alternative have disadvantages compared to the hash table?

\subsection*{Solution}

\textit{If all edge lookups are equally likely, what is the expected time to
determine whether an edge is in the graph?}

In practice, lookup time depends on the implementation of the hash table
(number of collisions and collisions handling). In theory, $O(1)$ is considered
as the expected time for hash table lookup.

\textit{What disadvantages does this scheme have?}

The hash table approach has the following disadvantages:

\begin{itemize}
	\item Discovery of all vertices $v$ which are connected to some vertex $u$
	(adjacency vertices) --- one would need to go through the whole hash table
	to find out this piece of information.

	\item Collisions handling --- depending on the size of the hash table and
	on the implementation of the hash function, it can happen that two vertices
	will eventually be mapped to the same place, so one need a mechanism to
	solve such collisions, which can degrade performance of hash table
	operations.

	\item Memory intensity --- hash tables are (in general) more memory
	intensive than other techniques (such as lists).
\end{itemize}

\textit{Suggest an alternate data structure for each edge list that solves
these problems.}

Instead of a hash table (and other techniques presented in the lecture), one
can use an \emph{associative array} implemented via some sort of a
\emph{self-balancing binary search tree} (such as an AVL tree or a red-black
tree). Using this data structure, one can easily determine all adjacent
vertices (one pass through the tree), there is no collisions handling needed
and the underlying tree needs space only for vertices (there is no overhead),
thus providing a solution to the memory issue of hash tables.

\textit{Does your alternative have disadvantages compared to the hash table?}

An associative array (as presented above) has the following disadvantages:

\begin{itemize}
	\item Slower insertion, lookup and deletion of vertices --- in constrast to
	hash tables, balanced binary trees need $O(log(n))$ time to perform these
	operations.
\end{itemize}

%%%%%%%%%%%%%%%%%%%%%%%%%%%%%%%%%%%%%%%%%%%%%%%%%%%%%%%%%%%%%%%%%%%%%%%%%%%%%%%
% vim: syntax=tex
%%%%%%%%%%%%%%%%%%%%%%%%%%%%%%%%%%%%%%%%%%%%%%%%%%%%%%%%%%%%%%%%%%%%%%%%%%%%%%%
