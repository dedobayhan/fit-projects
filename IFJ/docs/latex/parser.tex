\section {Syntaktická a sémantická analýza}

Syntaktická analýza (dále jen SA) je \uv{srdcem} našeho programu a skládá se ze dvou částí: \textbf{prediktivní SA} (použita pro kontrolu deklarací/definic, těla funkcí a volání funkce)
a \textbf{precedenční SA} (analýza výrazů). Principem je křížové volání těchto dvou částí v průběhu analýzy. Ze sémantických důvodů bylo nutné SA rozdělit na dvě fáze, a to deklarace (funkcí a globálních proměnných) a definice (tělo funkce, její parametry a lokální proměnné).

SA spolupracuje s lexikální analýzou a tabulkou symbolů (dále jen TS). Byla vytvořena speciální funkce, která volá lexikální analyzátor, provádí část sémantické kontroly a spolupracuje s TS (číst zdrojový text je nutné ve více částech SA).

Obě analýzy používají shodné struktury, které jsou implementovány dostatečně obecně. Pravá strana pravidel je implementována jako pole, usnadňuje to převrácení pravidla při vkládání na zásobník v prediktivní SA a kontrolu při redukci v precedenční SA.

\subsection{Prediktivní SA}

Tato metoda je založena na LL tabulce (která navíc obsahuje speciální pravidlo, které značí, že následuje výraz a tudíž budeme volat precedenční SA na jeho zpracování).
Volání funkce řeší právě tato metoda - proto není např. nutné v tělě funkce kontrolovat, zda příkaz začíná voláním funkce (normálně by se musela volat precedenční SA), nebo se jedná o výraz (syntaktická chyba)

Vlastní algoritmus používá zásobník na terminály a non-terminály se standardními operacemi (push, top, pop, ...) - implementován jednosměrně vázaným seznamem.

Je nutné uchovávat některé údaje, např. poslední použité pravidlo, poslední načtený identifikátor, jeho datový typ a index na datovém zásobníku při interpretaci (protože prediktivní SA při nalezení shody terminálu na zásobníku s aktuálně vráceným tokenem provádí odstranění tohoto terminálu ze zásobníku, což pro nás znamená ztrátu informace).

Při volání funkce se ve spolupráci s TS kontroluje, zda sedí datové typy argumentů s deklarací funkce. Generování příslušných instrukcí probíhá v průběhu SA podle použitých pravidel a posloupnosti načtených tokenů.

\subsection{Precedenční SA}

Hlavní část této analýzy je precedenční tabulka, která obsahuje (podobně jako prediktivní metoda) speciální pravilo, které značí volání funkce (a tudíž volání prediktivní SA na zpracování) - z tohoto důvodu nám stačí pouze jeden non-terminál a nemusíme zavádět speciální pravidla pro \uv{nový operátor čárka} (použitý při volání funkce jako oddělovač argumentů).

V algoritmu je použit speciálně upravený zásobník pro potřeby této analýzy (implementován obousměrně vázaným seznamem), který obsahuje (mimo základní operace nad zásobníkem) tyto operace:
\begin{itemize}
\item vrať terminál nejblíže vrcholu zásobníku
\item vrať non-terminál nejblíže vrcholu zásobníku (při návratu z analýzy má tento non-terminál atributy výsledku - vrácen do prediktivní SA, kde jsou tyto údaje následně využity při sémentické kontrole a nagenerování instrukcí)
\item \uv{shift} - vložení pomocné položky před nejvrchnější terminál (analogie $<$ z přednášek)
\item \uv{redukce} - záměna pravé strany pravidla (tzv. handle) na zásobníku (analogie $>$ z přednášek)
\end{itemize}

Non-terminály obsahují atributy - datový typ a index na datovém zásobníku, který bude použit při interpretaci. Toho je využíváno při generování instrukcí (k tomu dochází při redukci na zásobníku), protože interpret již nezná konkrétní datové typy a řídí se pouze indexy. Při sémantické kontrole dochází ke kontrole typů a případným konverzím (podle typu operace).
