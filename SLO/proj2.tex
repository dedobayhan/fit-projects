%
% Encoding: utf-8
% Author:   Petr Zemek, 2010
%
\documentclass[11pt,a4paper]{article}
\usepackage[a4paper, top=1.5cm, bottom=1.5cm, right=1.5cm, left=1.5cm, nohead]{geometry}
\usepackage[utf8]{inputenc}
\usepackage[czech]{babel}
\usepackage{amsmath, amsthm, amssymb}

% Paragraph formatting - no indents
\setlength{\parskip}{1.3ex plus 0.2ex minus 0.2ex}
\setlength{\parindent}{0pt}

% Theorems
\newtheorem*{theorem}{Věta}

% Commands
\newcommand{\st}[0]{\,|\,} % So that...

\begin{document}

% Date
\begin{flushright}
	\today
\end{flushright}

% Title
\begin{center}
	\begin{large}\textbf{2. úkol z~předmětu Složitost}\end{large} \\
	\vspace{0.4cm}
	Petr Zemek \\
	\textit{xzemek02@stud.fit.vutbr.cz} \\
	\textit{Fakulta Informačních Technologií, Brno} \\
\end{center}

\section*{Příklad 1}

Nechť $n$ značí počet uzlů v grafu. Složitost tohoto algoritmu je závislá nejen na vstupu, ale také na implementaci prioritní fronty, $Q$. V zadání je uvedeno, že odhady máme uvést vzhledem k počtu uzlů v grafu, tak budu předpokládat, že tato fronta je implementována pomocí pole o velikosti $n + 1$, kde první prvek určuje počet prvků ve frontě (využívá se ho pro testování neprázdnosti fronty na řádku 7; na řádku 6 se jeho hodnota nastavuje na $n$ a na řádku 11 dochází k jeho snížení o 1).

\subsection*{Analýza časové složitosti}

Řádky 3--5, 9--10, a 13--17 mají konstantní časovou složitost $O(1)$. Řádek 6 má lineární časovou složitost $O(n)$ (inicializace pole o $n + 1$ prvcích). Řádky 7 a 11 mají pro danou implementaci konstantní časovou složitost $O(1)$ (testování hodnoty či snížení hodnoty jednoho čísla). Řádek 8 má pro danou implementaci lineární časovou složitost $O(n)$ (prochází se polem $Q$ a hledá se uzel $u$ s nejmenší hodnotou $dist[u]$).

Cykly na řádcích 2--4 a 7--16 se provedou oba $O(n)$-krát (pro každý uzel jedna iterace). Cyklus na řádcích 12--16 se provede také $O(n)$-krát (v nejhorším případě je graf úplný, tj. každý uzel je spojen s každým uzlem).

Celková časová složitost je tedy $O(n)$ (řádky 2--5) $+\ O(n)$ (řádek 6) $+\ O(n^{2})$ (řádky 7--17), což je $O(n) + (n) + O(n^{2}) = O(n^{2})$.

\subsection*{Analýza prostorové složitosti}

V algoritmu se používají dvě pole o velikosti $n$ ($dist$ a $previous$), jedno pole o velikosti $n + 1$ (fronta $Q$) a konstantní počet pomocných proměnných ($u$, $v$, $alt$). Jelikož do prostorové složitosti nemáme počítat velikost vstupu, tak je celková prostorová složitost algoritmu $O(n)$.

\section*{Příklad 2}

Symbol '$+$' znamená, že daná příslušnost platí; symbol '$-$' naopak značí, že daná příslušnost neplatí.

\begin{center}
	\begin{tabular}{l|c|c|c}
			& $f(n) \in O(g(n))$ & $f(n) \in \Omega(g(n))$ & $f(n) \in \Theta(g(n))$ \\
		\hline
		(a) & $+$ & $+$ & $+$ \\
		(b) & $+$ & $-$ & $-$ \\
		(c) & $-$ & $+$ & $-$ \\
		(d) & $+$ & $-$ & $-$
	\end{tabular}
\end{center}

\section*{Příklad 3}

\begin{theorem}
	$\textbf{P} = \textbf{NP} \Longrightarrow \textbf{NP} = \textbf{co-NP}$
\end{theorem}

\begin{proof}
	Triviálně vyplývá z faktu, že třída \textbf{P} je uzavřena vůči doplňku.
\end{proof}

\end{document}
